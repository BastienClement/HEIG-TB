\chapter{Définition théorique des signaux}
	
Les signaux sont des constructions fonctionnelles semi-pures. Puisqu'ils encodent la notion de variabilité, ils sont naturellement dépendant du temps en tant qu'état global. Plus spécifiquement, ils dépendent d'un \emph{indice de génération} spécifique à ce signal désigné par $\gamma_s \in \Gamma_s$ qui est associé à chaque changement potentiel d'état du signal.

Dans le cas d'un signal enfant, l'indice de génération est défini comme un $n$-uplet constitué des indices de générations de chaque signaux parents sur lesquels il est dépendant. De cette façon, l'indice et donc l'état des signaux parent est encodé dans l'indice du signal enfant et la dépendance vers d'autres signaux ne compromet pas sa pureté:
\[
	\mathbb{D}^\gamma_{child} = \{ a, b, \dots \} \implies \Gamma_{child} = \Gamma_a \times \Gamma_b \times \dots
\]
avec $\mathbb{D}^\gamma_{a}$ l'ensemble des signaux envers lesquels $a$ est dépendant pour l'indice de génération \gamma. Par définition $|\mathbb{D}^\gamma_{a}| > 0 \iff a$ est un signal enfant.

Dans le cas d'un signal source, l'indice de génération est une valeur abstraite et distincte pour chaque changement d'état.
\[
	\mathbb{D}^\gamma_{source} = \emptyset \implies \forall \alpha \in \Gamma_{source}, \forall \beta \in Signals, (\alpha \notin \Gamma_\beta) \lor (\beta = source)
\]
Un signal peut alors être considéré comme une fonction pure $\Gamma_{sig} \to State_{[T]}$ associant à un indice de génération spécifique un état précis dont la valeur, si elle est définie, est de type $T$.
\[
	sig(\gamma_{sig})\colon \Gamma_{sig} \to State_{[T]} = state \in \{ Undefined, Defined_{[T]}(value) \}
\]
Les changements d'état d'un signal sont des événements séquentiels. L'ensemble des indices de génération de ce signal forment ainsi un ensemble ordonné sur lequel il est possible de définir la fonction
\begin{align*}
	\gamma^*_{sig} = prev(\gamma_{sig}) &\colon \Gamma_{sig} \to \Gamma_{sig}\\
	& = \begin{cases}
		\gamma^*_{sig} & \text{si }
		 \exists \gamma^*_{sig}, \forall \gamma^\alpha_{sig} < \gamma_{sig}, (\gamma^\alpha_{sig} \leq \gamma^*_{sig})\\
		\varnothing & \text{sinon}
	\end{cases}
\end{align*}
associant à chaque indice $\gamma_{sig}$ l'indice $\gamma^*_{sig}$ associé à l'état qui précédait immédiatement l'état associé à l'indice $\gamma_{sig}$. Cette propriété autorise un signal à être défini non seulement en fonction des états actuels d'autres signaux, mais aussi de leurs états antérieurs.

\begin{figure}
	\begin{align*}
		hold (sig) &\colon Signal_{[T]} \to Signal_{[T]} \\
          &\colon (\Gamma_{sig} \to State_{[T]}) \to \Gamma_{sig} \to State_{[T]} \\
          &= \gamma \mapsto \begin{cases}
		           		sig(\gamma) & \text{si } sig(\gamma) \text{ est défini} \\
		           		hold(sig)(\gamma^*) & \text{sinon si } \gamma^* \ne \varnothing \\
		           		Undefined & \text{sinon}
          \end{cases}
	\end{align*}
	\caption{Définition de la fonction $hold$}
	\label{fig:sig-hold}
\end{figure}

\begin{figure}
	\begin{lstlisting}
def hold[T](parent: Signal[T]): Signal[T] = {
var previous: Option[T] = None
Signal {
	val state = parent.option
	if (state.isDefined) previous = state
	state orElse previous
}
}
	\end{lstlisting}
	\caption{Implémentation de la fonction $hold$}
	\label{fig:sig-hold-scala}
\end{figure}

La figure \ref{fig:sig-hold} présente la définition d'une fonction $hold$ exploitant ce mécanisme pour construire un signal enfant qui maintient sa valeur lorsque son parent devient indéfini.

En pratique, le concept d'indice de génération est implicite. L'accès à un signal se fait toujours à partir de son état le plus récent et la disponibilité des états antérieurs est implémenté en utilisant des variables encapsulées dans le contexte de définition du signal. Les constructions ainsi formées ne sont donc pas strictement pures d'un point de vue fonctionnel mais sont compatible avec la notion théorique d'un signal et ne causent pas de surprise lors à l'usage. La figure \ref{fig:sig-hold-scala} présente une implémentation possible de la fonction $hold$ en Scala.

En résumé, les fonctions utilisées pour définir ou transformer un signal doivent être \emph{semi-pures}:
\begin{enumerate}
	\item elles ne peuvent dépendre d'aucun état implicite, hormis d'autres signaux et leurs états antérieurs, et
	\item elles ne doivent pas contenir d'effets de bords observables
\end{enumerate}
De cette façon, l'ordre d'évaluation des signaux ou même leur évaluation différée n'a pas d'importance dans le comportement du système. Ces propriétés ne sont pas vérifiables au niveau du langage, le développeur est ainsi responsable de s'assurer que ses fonctions soient conformes à ces contraintes.

Le concept d'\emph{observateur} (Section \ref{sec:sig-obs}) est un mécanisme permettant d'introduire des effets de bords à partir de signaux, de façon sûre et définie.

\section{Opérations élémentaires}

	\subsection{Sélection (flatMap)}
		
		\begin{align*}
			flatMap(a, f)
				&\colon (Signal_{[T]}, T \to Signal_{[U]}) \to Signal_{[U]} \\
				&\colon (\Gamma_a \to State_{[T]}, T \to \Gamma_f \to State_{[U]}) \to \Gamma_{a \times f} \to State_{[U]} \\
				&= \gamma \mapsto \begin{cases}
					f(v)(\gamma_f) & \text{si } a(\gamma_a) = Defined(v)\\
					Undefined & \text{sinon}\\
				\end{cases}
		\end{align*}
	
	\subsection{Application (map)}
		
		\begin{align*}
			map(a, f)
				&\colon (Signal_{[T]}, T \to U) \to Signal_{[U]} \\
				&\colon (\Gamma_a \to State_{[T]}, T \to U) \to \Gamma_a \to State_{[U]} \\
				&= flatMap \big( a, v \mapsto \gamma \mapsto Defined(f(v)) \big)\\
				&= \gamma \mapsto \begin{cases}
					Defined \big(f (v) \big) & \text{si } a(\gamma) = Defined(v)\\
					Undefined & \text{sinon}\\
				\end{cases}
		\end{align*}
	
	\subsection{Filtrage (filter)}
	
		\begin{align*}
			filter(a, p)
				&\colon (Signal_{[T]}, T \to Boolean) \to Signal_{[T]} \\
				&\colon (\Gamma_a \to State_{[T]}, T \to Boolean) \to \Gamma_a \to State_{[T]} \\
				&= flatMap \left( a, v \mapsto \gamma \mapsto \begin{cases}
					a(\gamma) & \text{si } p(v)\\
					Undefined & \text{sinon}\\
				\end{cases} \right)\\
				&= \gamma \mapsto \begin{cases}
					a(\gamma) & \text{si } a(\gamma) = Defined(v) \text{ et } p(v)\\
					Undefined & \text{sinon}\\
				\end{cases}
		\end{align*}
		
	\subsection{Lecture}
	
		\[
			a.option = \begin{cases}
				Some(a.value) & \text{si } a \text{ est défini}\\
				None & \text{si } a \text{ est indéfini}
			\end{cases}
		\]
		\[
			a.value = \left(opt \mapsto \begin{cases}
				value & \text{si } opt \text{ est } Some(value)\\
				Nothing & \text{si } opt \text{ est } None
			\end{cases}\right) \circ a.option
		\]
