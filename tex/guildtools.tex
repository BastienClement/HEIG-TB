\chapter{GuildTools}

\textit{Note: Cette section décrit très brièvement l'application GuildTools qui sera développée en utilisant le framework réactif-fonctionnel. La liste de fonctionnalités correspond aux fonctionnalités prévues au moment de ce rapport intermédiaire. Ces fonctionnalités peuvent être amenées à varier d'ici au rendu final. }

\section{Objectif}

Application de gestion de guilde sur World of Warcraft.

\section{Fonctionnalités}

	\subsection{Profil de joueur}

	Ce premier module est destiné à collecter et gérer l'ensemble des données de profil d'un joueur: son nom et prénom, son age, ses informations de contact en jeu ainsi que ses différents personnages.

	\subsection{Calendrier}
	
	Un calendrier partagé permettant de définir les soirées de jeu de l'équipe. 
	
	Chaque événement offre une liste des joueurs enregistrés comme présents ou absents, ainsi qu'une note optionnelle permettant au joueur d'apporter des précisions supplémentaires.
	
	Un système de gestion des absences de façon globale est aussi disponible, permettant aux joueurs de définir des plages pendant lesquels ils sont indisponibles à tous les événements créés ou futurs.
	
	\subsection{Roster}
	
	Une vue d'ensemble des joueurs du groupe et de leurs personnages avec la possibilité de filtrer selon différents critères.
	
	\subsection{Whishlists}
	
	Un module permettant de saisir les besoins des différents personnages d'un joueur, facilitant ainsi la composition des groupes par les officiers.
	
	\subsection{Composition de groupes}
	
	Un module simplifiant la construction de groupe de jeu en s'assurant qu'un joueur ne se trouve pas simultanément dans plusieurs groupes. Ces groupes sont ensuite exportables vers un événement calendrier.
	