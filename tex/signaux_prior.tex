\section{Solutions existantes}

\subsection{ReactiveX}
Dans les implémentations les plus populaires de framework réactifs-fonctionnels, dont notamment la bibliothèque \emph{ReactiveX} disponible pour de nombreux langages, on retrouve généralement un concept similaire sous le nom de \texttt{Rx}.

Bien que tous deux soient une abstraction du temps, les \texttt{Rx} représentent fondamentalement une séquence tandis qu'un \texttt{Signal} est une valeur unique à un instant donné. Afin de clarifier cette distinction, ces deux interfaces peuvent être comparées à leur équivalent non-réactif le plus proche dans la bibliothèque standard de Scala:

\begin{table}[H]
	\begin{tabular}{@{}p{2.5cm}p{2.5cm}p{\dimexpr\textwidth-6cm\relax}@{}}
		\toprule
		Type de base & Type réactif & Différence \\ \midrule
		\texttt{Stream[T]} & \texttt{Rx[T]} & Tous les éléments d'un \texttt{Stream} sont calculables à l'instant présent, les éléments d'une \texttt{Rx} ne sont peut être pas encore disponibles \\
		&  & \\
		\texttt{Future[T]} & \texttt{Signal[T]} & Un \texttt{Future} n'est résolu qu'une seule fois, un \texttt{Signal} peut changer de valeur de multiples fois au cours du temps \\ \bottomrule
	\end{tabular}
\end{table}

Ainsi, certaines opérations monadiques, dont la sémantique peut être ambiguë dans le cas des valeurs réactives, peuvent présenter des comportements considérablement différents entre leur implémentation par les signaux et les \texttt{Rx}s. C'est le cas notamment de la fonction \texttt{flatMap}
\footnote{\texttt{Signal[T].flatMap[U](fn: T => U): Signal[U]}, de façon similaire pour \texttt{Rx[T]}}:

\begin{itemize}
	\item Sur une valeur \texttt{Rx}, l'opération \texttt{flatMap} effectue une concaténation des valeurs des sous-séquences retournées, potentiellement entrelacées.
	\item Sur un \texttt{Signal}, l'opération \texttt{flatMap} se comporte comme un switch: le signal original est utilisé pour sélectionner un second signal, dont la valeur sera celle du signal produit par \texttt{flatMap}. Lorsque la valeur du signal initial change, la sélection est effectuée à nouveau et le signal résultant est utilisé comme nouvelle source de valeurs.
\end{itemize}

Bien que la bibliothèque \emph{ReactiveX} propose un opérateur spécifique avec la sémantique de switch, il est utile de préciser que l'opération \texttt{flatMap} est l'opération utilisée par le langage Scala lors de l'utilisation de multiples générateurs dans une compréhension \texttt{for}.

Ainsi le code
\begin{lstlisting}
val a: Signal[Int] = ...
val b: Array[Signal[Double]] = ...
val c: Signal[Double] = for (x <- a; y <- b(x)) yield y * 2
\end{lstlisting}

sera transformé par le compilateur en
\begin{lstlisting}
val c = a.flatMap(x => b(x)).map(y * 2)
\end{lstlisting}

et produira des résultats radicalement différents selon l'abstraction utilisée:

\begin{itemize}
	\item Dans le cas des \texttt{Rx}, le résultat est l'agglomération des tous les flux de valeurs de \texttt{b} sélectionnés au fil du temps, potentiellement de multiples fois. Le résultat n'est très probablement pas celui escompté.
	
	\item Dans le cas de \texttt{Signal}, le résultat est un signal \texttt{c} dont la valeur est le double de celle d'un signal contenu dans le tableau \texttt{b} et désigné par la valeur du signal \texttt{a}, le tout sur la base des valeurs de ces signaux au moment présent. Dans le futur, lorsque la valeur de \texttt{a} ou de \texttt{b(x)} changera, la valeur de \texttt{c} sera également mise à jour.
\end{itemize}

La sémantique des signaux, basée sur les valeurs plutôt que la séquence de ces valeurs, a pour but d'être la plus adaptée possible à l'usage qui en est fait dans ce travail, c'est à dire la conception d'interfaces graphique.

Une conséquence supplémentaire de cette différence entre signaux et \texttt{Rx} est la possibilité de définir aisément le concept de \emph{signal expression} dans le cas des signaux. Dans le cas des \texttt{Rx}, le comportement attendu d'une telle construction n'est pas évident dans une situation où les flux sont potentiellement finis.

\subsection{Scala.rx} \label{sec:sig-comp-scalarx}
Scala.rx est l'implémentation la plus proche des signaux utilisés dans ce projet. La simplicité d'utilisation est particulièrement appréciable avec la méthode de définition de \texttt{Rx}s sur la base d'expressions arbitraires.

Les différences sont principalement liées aux limitations volontairement imposées dans Scala.rx.

\subsubsection{Pas de variable réactive non-définie / vide}
La raison invoquée est la difficulté d'intégrer l'absence de valeur de façon intuitive dans la syntaxe déclarative. Plus spécifiquement, l'auteur mentionne plusieurs solutions potentielles:
\begin{enumerate}
	\item Bloquer le thread courant jusqu'à ce qu'une valeur soit disponible. Cette solution n'est cependant pas envisageable pour une implémentation visant l'environnement Javascript.
	\item Lancer une exception lors de l'accès à une variable vide, mais recommencer l'opération une fois que sa valeur sera définie. Cette option présente le désavantage que le calcul d'une variable réactive puisse potentiellement être démarré puis interrompu de multiples fois si de nombreuses dépendances sont indéfinies.
	\item Obliger le style monadique et retirer la méthode implicite de définition. Cette option n'est pas considérée pour des raisons d'expérience utilisateur évidentes.
	\item Utiliser un plugin du compilateur pour transformer automatiquement le code en style monadique. Cette approche présente cependant de nombreux défis techniques et requiert une étape supplémentaire de configuration indésirable.
\end{enumerate}

En pratique l'absence de variable réactive vide est une gêne considérable dans le domaine du web où toutes les opérations sont asynchrones et non-bloquante. Le concept de promesse est omni-présent et il est ainsi difficile d'associer les interfaces natives du navigateur avec le réseau de variables réactives.

Une première approche est d'utiliser directement le type \texttt{Option} de Scala pour exprimer l'absence d'information le temps de l'opération. Mais cette idée impose une verbosité beaucoup plus importante du code puisqu'il est maintenant nécessaire de manipuler explicitement des \texttt{Rx[Option[T]]} au lieu de simples \texttt{Rx[T]} et de laisser le soin de la gestion de l'asynchrone à la bibliothèque.

De plus, la seconde approche proposée (interrompre le calcul par une exception) ne semble pas déraisonnable et peut constituer une approche valide si correctement documentée. Il est important que les conséquences d'une variable indéfinie soient connues et considérées, mais leur absence volontaire semble au final plus gênante que bénéfique. Par ailleurs, le style monadique ne présente pas les inconvénients mentionnés et constitue une alternative valide si le style est plus adapté à un problème spécifique.

\subsubsection{Complexité de l'implémentation}
Malgré l'effort fourni pour proposer une interface simple et efficace, l'implémentation est en réalité relativement complexe et très largement basée sur l'utilisation de macros. Le code écrit par le développeur n'est pas le code finalement exécuté. Une raison probable de cette implémentation est basée sur la méthode choisie pour détecter automatiquement les dépendances entre variables réactives avec le style déclaratif.

Dans Scala.rx, les expressions qui définissent des variables réactives sont réécrite sous formes de fonctions exploitant largement le système de paramètres implicites pour transmettre le contexte d'accès à une variable réactive entre parent et enfant. Bien que cela soit une approche très proche des standards en Scala, elle requiert une couche syntaxique supplémentaire ou l'utilisation de macros afin de l'ajouter automatiquement. Le système résultant est excessivement complexe et les erreurs rencontrées lors de l'usage ne sont pas intuitives puisqu'elles ne correspondent pas au code écrit. 

De plus, l'approche n'est pas portable entre les versions de Scala puisqu'elle dépend de l'implémentation interne du compilateur. Dans le cas de Scala.rx, l'implémentation est compilée simultanément pour les version 2.10, 2.11 et 2.12 de Scala et requiert pour chaque version une implémentation différente des macros pour s'aligner avec les changements apportés au compilateur et à la syntaxe du langage.

Une approche alternative est possible: \texttt{DynamicVariable}. Cette classe de la bibliothèque standard de Scala est rarement utilisée puisque très spécifique et généralement moins explicite que l'usage de paramètres implicites qui sont plus en accord avec les principes de la programmation fonctionnelle. Elle remplit cependant un rôle semblable en offrant une sémantique de variable à portée dynamique plutôt que lexicale. Ce mécanisme permet de déplacer la transmission du contexte d'évaluation des paramètres implicites à un canal annexe dédié.

Le code est ainsi fortement simplifié puisque la transmission du contexte est clairement dissociée, évitant la nécessité de modifier le code utilisateur et donc l'usage de macros au prix d'une architecture pouvant être considérée moins pure d'un point de vue fonctionnel. L'absence de macros permet également une compatibilité plus aisée avec les futures versions du langage.

En contrepartie, la détection automatique des dépendances dépend de l'exécution synchrone de l'expression de définition (§~\ref{sec:sig-pureness}). De façon générale, effectuer le calcul de la valeur d'un signal de façon différée par rapport à l'instant d'évaluation choisi par la bibliothèque ne semble pas compatible avec les objectifs de fournir un modèle d'évaluation strict et prévisible.
