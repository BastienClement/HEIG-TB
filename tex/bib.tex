\renewcommand\bibname{Références}
\begin{thebibliography}{99}

\bibitem{russell2011}
Alex Russell.
\emph{Web Components and Model Driven Views by Alex Russell}.
Fronteers Conference 2011\\
\url{https://fronteers.nl/congres/2011/sessions/web-components-and-model-driven-views-alex-russell}

\bibitem{russellIN}
Alex Russell.
\emph{Infrequently Noted}.
Blog personnel\\
\url{https://infrequently.org/}

\bibitem{stateOfWebComp}
Wilson Page.
\emph{The state of Web Components}.
Mozilla Hacks blog\\
\url{https://hacks.mozilla.org/2015/06/the-state-of-web-components/}

\bibitem{odersky2012}
Ingo Maier et Martin Odersky, 2012.
\emph{Deprecating the Observer Pattern with Scala.React}.
EPFL-REPORT-176887\\
\url{https://infoscience.epfl.ch/record/176887}

\bibitem{scala-react}
Ingo Maier, 2012.
\emph{Scala.react is a reactive programming library for Scala}.\\
\url{https://github.com/ingoem/scala-react}

\bibitem{scala.rx}
Li Haoyi et al., 2012--2016.
\emph{scala.rx: An experimental library for Functional Reactive Programming in Scala}.\\
\url{https://github.com/lihaoyi/scala.rx}

\bibitem{haskell-monad-laws}
HaskellWiki.
\emph{Monad laws}.\\
\url{https://wiki.haskell.org/Monad_laws}

\bibitem{czaplicki}
Evan Czaplicki, 2012.
\emph{Elm: Concurrent FRP for Functional GUIs}.
\\
\url{https://www.seas.harvard.edu/sites/default/files/files/archived/Czaplicki.pdf}

\bibitem{czaplicki}
Conal Elliott, 2009.
\emph{Push-Pull Functional Reactive Programming}.
Haskell Symposium\\
\url{http://conal.net/papers/push-pull-frp}

\bibitem{polymer-project}
Polymer Project\\
\url{https://www.polymer-project.org/}

\bibitem{xml-ebnf}
W3C, 2008. \emph{Extensible Markup Language (XML) 1.0 (Fifth Edition)}, Notation\\
\url{https://www.w3.org/TR/REC-xml/#sec-notation}

\bibitem{w3c-shadowdom}
W3C, 2017. \emph{Shadow DOM} (Working Draft)\\
\url{https://www.w3.org/TR/shadow-dom/}

\bibitem{w3c-css-scopings}
W3C, 2014. \emph{CSS Scoping Module Level 1} (Working Draft)\\
\url{https://www.w3.org/TR/css-scoping-1/}

\bibitem{w3c-custom-elements}
W3C, 2016. \emph{Custom Elements} (Working Draft)\\
\url{https://www.w3.org/TR/custom-elements/}

\bibitem{google-shadowdom}
Google. \emph{Shadow DOM v1: Self-Contained Web Components} (Web Fundamentals)\\
\url{https://developers.google.com/web/fundamentals/getting-started/primers/shadowdom}

\bibitem{scalanative}
Scala Native\\
\url{http://www.scala-native.org}

\bibitem{llvm}
The LLVM Compiler Infrastructure\\
\url{https://llvm.org/}

\bibitem{typescript}
TypeScript, \emph{JavaScript that scales}\\
\url{https://www.typescriptlang.org/}

\bibitem{clojure-script}
ClojureScript\\
\url{https://clojurescript.org/}

\bibitem{emscripten}
Emscripten: \emph{An LLVM-to-JavaScript Compiler}\\
\url{https://github.com/kripken/emscripten}

\bibitem{closure-compiler}
Google, \emph{What is the Closure Compiler?}\\
\url{https://developers.google.com/closure/compiler/}

\bibitem{w3c-css-flexbox}
W3C, 2016. \emph{CSS Flexible Box Layout Module Level 1} (Candidate Recommendation)\\
\url{https://www.w3.org/TR/css-flexbox-1/}

\bibitem{w3c-css-contain}
W3C, 2017. \emph{CSS Containment Module Level 1} (Working Draft)\\
\url{https://www.w3.org/TR/css-contain-1/}

\bibitem{scalatags}
ScalaTags\\
\url{https://github.com/lihaoyi/scalatags}

\bibitem{scalacss}
ScalaCSS\\
\url{https://github.com/japgolly/scalacss}

\bibitem{angular}
Angular\\
\url{https://angular.io/}

\end{thebibliography}