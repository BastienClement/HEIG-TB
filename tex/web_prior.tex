\section{Solutions existantes}

\subsection{Bindings.scala et Monadic-HTML}

\emph{Bindings.scala} \cite{binding.scala} et \emph{Monadic-HTML} \cite{monadic-html} sont deux exemples d'implémentation d'un système de template basés sur les concepts de la programmation réactive-fonctionnelle.

Ces deux bibliothèques sont très similaires, \emph{Monadic-HTML} étant initialement inspirée de \emph{Bindings.scala}. Parmi les différences notables, on peut noter l'approche basée sur les macros compilateur utilisée par \emph{Bindings.scala} et la ré-implémentation des classes XML de Scala dans \emph{Monadic-HTML}.

Dans les deux cas, l'objectif est de transformer le code du développeur composé sur la base des littéraux XML intégrés au langage Scala en une structure dynamique qui peut être insérée dans l'arbre DOM du document.

Les littéraux XML, initialement ajoutés au langage Scala alors que l'usage de XML était beaucoup plus prévalent qu'il ne l'est aujourd'hui, sont aujourd'hui considéré comme une complexité excessive du langage et sont sur le point d'être retirés en faveur d'un interpolateur \texttt{xml}. Dans le cas de \emph{Bindings.scala}, l'implémentation sous forme de macros devra être ré-implémentée pour supporter la nouvelle syntaxe proposée.

Dans le cas de \emph{Monadic-HTML}, l'approche n'est tout simplement plus valide. La bibliothèque utilise en effet une implémentation alternative des classes XML utilisées par Scala pour matérialiser les littéraux XML à l'exécution, en principe disponibles dans une bibliothèque indépendante. Puisque la syntaxe n'est que sucre syntaxique au niveau du compilateur, les nouvelles classes dont la sémantique est considérablement différentes de l'implémentation originale sont utilisées aveuglément par le compilateur et conduisent au résultat désiré.

Le retrait du support de la syntaxe XML au niveau du compilateur Scala serait synonyme de retrait de l'implémentation des sugres syntaxiques sur lesquels  \emph{Monadic-HTML} est construit, sans offrir d'alternatives évidentes puisque le mécanisme était un \emph{hack} dès le début.

Dans le cadre du développement de Xuen, le choix a été fait de mettre de côté les littéraux XML, pour se concentrer sur une implémentation basée sur les interpolateurs. Par conséquent, une partie du travail précédemment effectué par le compilateur Scala au moment de la compilation est à présent effectué au \emph{runtime} dans le navigateur. La disponibilité du DOM et du parser HTML du navigateur est un avantage au niveau de l'implémentation, mais les performances s'en trouvent inévitablement impactées. L'approche n'exclut pas une utilisation future de macros pour effectuer la compilation du template au moment de la compilation du code.

Une autre différence notable entre ces bibliothèques et Xuen est le niveau d'encapsulation utilisé.  \emph{Monadic-HTML} et \emph{Bindings.scala} produisent des morceaux d'arbre DOM qui sont ensuite insérés dans le document, sans encapsulation particulière, notamment en ce qui concerne l'application de styles CSS. Xuen propose dès le départ une abstraction sous forme de \emph{composants} indépendants et isolés construit sur la base des derniers standards du web.

L'implémentation des signaux de ces deux bibliothèques diffère légèrement de l'implémentation qui en est fait dans Xuen. Les signaux sont manipulés explicitement de façon monadique , il n'existe pas de signaux pouvant représenter des expressions arbitraire. La notion de signaux indéfinis est également absente. Il est nécessaire d'encapsuler explicitement les données manipulées dans une instance de la classe \texttt{Option} si le signal peut potentiellement être indéfini, conduisant à une plus grand verbosité à l'utilisation.